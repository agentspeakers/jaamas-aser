\documentclass[11pt]{article}
\usepackage{times}
\title{Information Sheet for `Agent Programming in the Cognitive Era' (Viewpoint Paper)}
\date{}

\begin{document}
\maketitle

\noindent\textbf{What is the novelty and significance of the area being discussed in the context of autonomous agents and multiagent systems?}

The paper explores possible future approaches to the development of autonomous intelligent agents and multiagent systems. Specifically, it examines the claim that in the so-called `Cognitive Era', intelligent systems will be trained using machine learning techniques, rather than developed by human programmers, see, e.g, (Kelly and Hamm, 2013). The means by which agent systems are developed is a central concern of the AAMAS community, and, in particular, of the well-established agent programming subcommunity within AAMAS.

\medskip\noindent\textbf{Why is it timely to provide a viewpoint article in this way now?}

The claim that future intelligent systems will be trained using machine learning techniques rather than developed by human programmers has implications for future research directions in agent programming and the development of methodologies and tools for programming agents. It is particularly topical, given that the AAMAS 2019 conference will, for the first time, include a special track on `Engineering Multiagent Systems'.

\medskip\noindent\textbf{Explain how you have sought to ensure that your viewpoint is not merely an individual perspective with little sound basis, and provides a synthesis of current thinking in addition to your own perspective so that it provides a rigorous analysis of the area.}

We provide a brief survey of the last 20 years of agent oriented programming and agent development platforms. We also survey in some detail previous work on integrating AI components and techniques into the BDI agent architecture and execution cycle. Our discussion of open research problems and possible future research directions for agent programming is grounded in this analysis.

\medskip\noindent\textbf{Have research, survey or viewpoint papers on the same research area been published and, in particular, have any been published recently?}

To the best of our knowledge, no papers have addressed this research area. 
%
The most closely related recent papers we are aware of are:

\emph{K. Kindriks (2014). The Shaping of the Agent-Oriented Mindset - Twenty Years of Engineering {MAS}, Engineering Multi-Agent Systems - Second International Workshop, {EMAS}
               2014, 1--14, Springer}

\emph{B. Logan (2018). An Agent Programming Manifesto. International Journal of Agent-Oriented Software Engineering, 6(2), 187-210.}

\medskip\noindent\textbf{What are the main differences between your viewpoint and these previously published papers?}

The (Hindriks 2014) paper traces the development of the agent oriented approach to programming from the 1990s to the present day. It briefly discusses AI-related topics but does not provide an analysis of the role of machine learning or AI in the future of agent programming research as in the present paper.
%
The (Logan 2018) paper presents an analysis of the limited adoption of agent programming techniques by the mainstream programming community. While it mentions the possible role of AI in the future development of agent oriented programming, it does not survey previous work in this area or offer a detailed discussion of open research problems and future research directions for the integration of AI into the BDI model as the current paper does. It does not consider the question of whether agents can trained rather than programmed.

\medskip\noindent\textbf{Have you published parts of your paper before?}

No.

\end{document}