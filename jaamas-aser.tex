%%%%%%%%%%%%%%%%%%%%%%% file template.tex %%%%%%%%%%%%%%%%%%%%%%%%%
%
% This is a general template file for the LaTeX package SVJour3
% for Springer journals.          Springer Heidelberg 2010/09/16
%
% Copy it to a new file with a new name and use it as the basis
% for your article. Delete % signs as needed.
%
% This template includes a few options for different layouts and
% content for various journals. Please consult a previous issue of
% your journal as needed.
%
%%%%%%%%%%%%%%%%%%%%%%%%%%%%%%%%%%%%%%%%%%%%%%%%%%%%%%%%%%%%%%%%%%%
\RequirePackage{fix-cm}
%
%\documentclass{svjour3}                     % onecolumn (standard format)
%\documentclass[smallcondensed]{svjour3}     % onecolumn (ditto)
\documentclass[smallextended]{svjour3}       % onecolumn (second format)
%\documentclass[twocolumn]{svjour3}          % twocolumn/Users/aricci/Library/Containers/com.apple.mail/Data/Library/Mail Downloads/7D7AC24F-556B-42D5-8F98-A614FEC67763/Relazione SE_IoT.pdf
%
\smartqed  % flush right qed marks, e.g. at end of proof
%
\usepackage{graphicx}
\usepackage[T1]{fontenc}
\usepackage{mathptmx}      % use Times fonts if available on your TeX system
\usepackage{courier}
%
% insert here the call for the packages your document requires
%\usepackage{latexsym} etc.
\usepackage{amsmath}
\usepackage{amssymb}
\usepackage{amsfonts}
\usepackage{alltt}
\usepackage{xspace}
\usepackage{natbib}

\usepackage{xcolor}
%%RHB: Commented this out as it seems to generate problems and isn't needed, I assume the JAAMAS guidelines don't ask for this to be added?
%\usepackage[justification=centering]{caption}
\usepackage{url}
\usepackage{booktabs}
\usepackage{bcprules}
\typicallabel{ClrInt$_4$}
\usepackage{ASL-ASC-JAIR}

%
% please place your own definitions here and don't use \def but
% \newcommand{}{}
\newcommand{\ie}[0]{i.e.,\xspace}
\newcommand{\eg}[0]{e.g.,\xspace}
\newcommand{\etc}[0]{etc.\xspace}

\newcommand{\code}[1]{\texttt{#1}}

\newcommand{\aser}{\textsf{AgentSpeak(ER)}}
\newcommand{\asl}{\textsf{AgentSpeak(L)}}
\newcommand{\jason}{\textsf{Jason}}
\newcommand{\AandA}{\textsf{A\&A}}
\newcommand{\aanda}{\textsf{A\&A}}
\newcommand{\moise}{\textsf{MOISE}}
\newcommand{\cartago}{\textsf{CArtAgO}}
\newcommand{\jacamo}{\textsf{JaCaMo}}

\newcommand{\galt}{\alt~}

%%%%%%%%%%%%%%%%%%%%%%%%%%%%%%%%%%%%%%%%%%%%%%%%%%%%%%%%%%%%%%%%%%%%%
\newcommand{\commenttoby}[3]{\sm{#1}{#2 \textbf{--#3}}}
\newcommand{\FORGET}[1]{}
\newcommand{\tba}[1]{\color{blue} TO BE ADDED: #1 \color{black}}
\newcommand{\ale}[1]{\color{blue} [ALE]: #1 \color{black}}
\newcommand{\more}{\color{blue} [...] \color{black}}
\newcommand{\tbc}[1]{\color{red} TO BE CLARIFIED: #1 \color{black}}
\newcommand{\todo}[1]{\color{red} TODO: #1 \color{black}}
\newcommand{\tbd}[1]{\color{orange} TO BE DISCUSSED: #1 \color{black}}
\newcommand{\FIXME}[1]{{{\bf \color{red}{#1}}}}
%%%%%%%%%%%%%%%%%%%%%%%%%%%%%%%%%%%%%%%%%%%%%%%%%%%%%%%%%%%%%%%%%%%%%%


\sloppy

%
% Insert the name of "your journal" with
% \journalname{myjournal}
%
\begin{document}

\title{Improving Plan and Intention Encapsulation in BDI Agent Programming -- The AgentSpeal(ER) Proposal 
%\thanks{Grants or other notes
%about the article that should go on the front page should be
%placed here. General acknowledgments should be placed at the end of the article.}
}
%\subtitle{Do you have a subtitle?\\ If so, write it here}

%\titlerunning{Short form of title}        % if too long for running head

\author{}

%\authorrunning{Short form of author list} % if too long for running head

%\institute{F. Author \at
%              first address \\
%              Tel.: +123-45-678910\\
%              Fax: +123-45-678910\\
%              \email{fauthor@example.com}           %  \\
%%             \emph{Present address:} of F. Author  %  if needed
%           \and
%           S. Author \at
%              second address
%}

\date{Received: date / Accepted: date}
% The correct dates will be entered by the editor


\maketitle

\begin{abstract}
\todo{OLD VERSION}
In this paper, we introduce AgentSpeak(ER), an extension of the AgentSpeak(L) language tailored to support encapsulation. 
%
The AgentSpeak(ER) extension allows for significantly improving the style of BDI agent programming along relevant aspects, including program modularity and readability, failure handling, and reactive as well as goal-based reasoning. 
%
The paper introduces the novel language, formalises the changes in the usual semantics of AgentSpeak, illustrate the features of the language through examples, and shows results of experiments evaluating the proposed language.
\end{abstract}

\section{Introduction}
\label{sec:intro}

\todo{[OLD VERSION]}

%
% About AgentSpeak(L)
%
{\asl} has been introduced in \cite{Rao96}  with the purpose of defining an expressive, 
abstract language capturing the main aspects of the Belief-Desire-Intention architecture~\cite{Bratman88,Georgeff:1987:RRP:1863766.1863818}, featuring a formally defined semantics and an abstract interpreter.
%
The starting point to define the language were real-world implemented systems, namely the
Procedural Reasoning System (PRS)~\cite{Ingrand:1992:ARR:629535.629890} and the Distributed Multi-Agent Reasoning System (dMARS).
%

%
% Concrete APL
% 
{\asl}  and PRS have become a main reference for implementing concrete Agent Programming Languages 
based on the BDI model: 
%
main examples are Jason~\cite{jason06,bordini:07} and ASTRA~\cite{DBLP:conf/prima/CollierRL15}.
%
Besides Agent Programming Languages, the {\asl} model has been adopted
as the main reference to development several BDI agent-based
frameworks and technologies~\cite{BordiniMAPlpa,BordiniMAPlta} as well
as serving as inspiration for theoretical work aiming to formalise
aspects of BDI agents and agent programming
languages~\cite{DBLP:conf/kr/WinikoffPHT02,DBLP:conf/promas/DennisFBFW07,DBLP:journals/aamas/BordiniFVW06}.


%
% Contribution
%
Existing Agent Programming Languages extended the language with
constructs and mechanisms making it practical from a programming point
of view~\cite{jason06}.
%
Besides, proposals in literature extended the model is order to make
it effective for specific kinds of systems --- e.g., real-time
systems~\cite{Vikhorev:2011:APP:2030470.2030529} -- or to
improve the structure of programs, e.g. in terms of
modularity~\cite{Madden2010,Nunes2014}.
%%RHB: Not sure what should go in XXX above?

%
Along this line, in this paper we describe a novel extension of the
{\asl} model --- called {\aser} --- featuring \emph{plan
  encapsulation}, i.e. the possibility to define plans that fully
encapsulate the strategy to achieve the corresponding goals,
integrating both the pro-active and the reactive behaviour.
%
% Key points
%
This extension turns out to bring a number of important benefits to
agent programming based on the BDI model, namely:
%
\begin{itemize}
\item improving the overall readability of the agent source code,
  reducing fragmentation and increasing modularity;
\item promoting a more goal-oriented programming style, enforcing yet
  preserving the possibility to specify purely reactive behaviour,
  properly encapsulated into plans for goals;
\item improving intention management, enforcing a one-to-one relation
  between intentions and goals --- so every intention is related to a
  (top-level) goal;
\item improving failure handling, in particular making it easier the
  management of failures related to plans for reacting to environment
  events;
\end{itemize}
%
% \noindent Besides the benefits in terms of agent programming, the approach reduces the gap between the design level and the programming level,  ...
% \item facilitate goal-based reasoning -- ...
% \end{itemize}

\noindent The remainder of the paper is organised as follows:
%
first we describe in details the motivations that lead to the proposal
of a new AgentSpeak extension (Section~\ref{sec:motivation});
%
then, we introduce and discuss {\aser}, defining the main concepts,
syntax and semantics --- first informally (Section~\ref{sec:proposal})
and then providing the formalisation of some key aspect
(Section~\ref{sec:formalisation}).
%
We discuss then the results of a first evaluation that has been
carried out, based on a prototype implementation extending the ASTRA
platform (Section~\ref{sec:evaluation}).
%
We conclude the paper discussing related work
(Section~\ref{sec:related}) and sketching future work
(Section~\ref{sec:conclusion}).



\section{Background}
\label{sec:background}

\todo{
- Background on role of Plan and Intentions in BDI Agent Programming \\
- The AgentSpeak(L) case \\
- Issues \\
}


\todo{EMAS VERSION}

The main motivation behind {\aser} comes from the experience using
agent programming languages based on the {\asl} model, {\jason} and
ASTRA in particular.
%%
Yet, these issues are relevant for any language based on the BDI
architecture.
%%
%The first issue is about the plan model.

% \subsection{Plan encapsulation}
%
In the BDI model, plans are meant to specify some means by which an
agent can satisfy an end~\cite{Rao96}.
%
In {\asl}, a plan consists of a rule of the kind \textsf{e : c <- b}.
%
The head of a plan consists of a triggering event \textsf{e} and a
context \textsf{c}.
%
The triggering event specifies why the plan was triggered, i.e., the
addition or deletion of a belief or goal.
%
In the following, we refer to plans triggered by event goals
as \emph{g-plans}, and plans triggered by
belief change (including percepts) as \emph{e-plans}.
%
The context specifies those beliefs that should hold given the agent's
current belief base if the plan is to be triggered.
%
The body of a plan is a sequence of actions or (sub-)goals.
%

%
% THE PROBLEM
%
In this approach --- as well as in planning, in general --- the
\emph{means} to achieve a goal (i.e., the plan body) is meant to be
fully specified in terms of the actions the agent should execute and
the (sub-)goals the agent should achieve or test.
%
In practice, when programming such systems, it is often the case that
the strategy (the means) adopted to achieve some goal (the end)
naturally includes reactions --- i.e., reacting to events asynchronously
perceived from the environment, including changes in the beliefs.
% 
This reflects more than just the ability of an agent to change/adapt
its course of actions; it allows the integration of reactivity as a
core ingredient of the strategy to achieve some goal.
%
This revised notion of a plan is not just a programming feature; it
also occurs naturally in human activity. For example, a fisherman with
the goal of catching fish waits for the event of a tug on their line
indicating a fish is on the hook. Reactivity is a key ingredient of many
activities that we perform to achieve specific goals, not only to
handle events that represents errors or unexpected situations (for the
current courses of actions).
%
It follows naturally that this is also an opportunity to extend the
plan model so as to fully \emph{encapsulate} reactions that are
part of the strategy to achieve the goal, as well as the subgoals that
are specific to that particular goal.

The use of e-plans to achieve goals is actually an important
conceptual brick of the {\asl} model.
%
Let us consider the robot cleaning example used to describe plans
in~\cite{Rao96}. 
%
One of the plans is:

\begin{small}
\begin{verbatim}
+location(waste,X) : location(robot,X) & location(bin,Y)
  <- pick(waste); !location(robot,Y); drop(waste).
\end{verbatim}
\end{small}

\noindent That is, as soon as the robot perceives that there is
waste at its location, then it can pick it up and bring it to the bin.
%
This e-plan is an essential brick of the overall strategy to achieve
the goal of cleaning the environment. The problem here is that it is
an implicit rather than explicit goal of the agent (since it is an e-plan, 
it is executed regardless of the agent currently having the goal of keeping the
environment clean). In practice, we adopt a maintenance
goal~\cite{Duff:2006:PMG:1160633.1160817} to clean the environment,
which includes reacting to cleaning up waste when we see it. In the
above program, this notion cannot be represented and remains in the
mind of the programmer/designer; as there is no g-plan for it, there
is no explicit trace in the agent mental structures about this
goal.

This problem can also be illustrated with the following
scenario. Consider an agent that includes a set of plans (a module
written by a third party) to handle social obligations. The module has
several e-plans for different types of obligations:
\begin{small}
\begin{verbatim}
+obligation(Ag,committed(Goal)) : .my_name(Ag)      <- ...
+obligation(Ag,achieve(Goal))   : .my_name(Ag)      <- ...
-obligation(Ag,Goal) : .my_name(Ag) & .intend(Goal) <- ...
\end{verbatim}
\end{small}
If for some reason during its execution an agent decides not to follow
these plans anymore (e.g.\ it chooses to become disobedient), it is
difficult to ``disable'' the behaviour of the above plans. Either
these plans have to be changed to consider a particular state of the
agent or the agent removes all these plans from its plan
library. Neither option is simple to
program. % suitable: the third party code could be not modified; we do not have references for these plans to remove them.
Although we can solve the problem, the lack of an explicit goal
stating that the agent intends to be obedient is the cause of this
problem.

%
%
%
Besides maintenance goal, also for achievement goals we see benefits in
encapsulating the reactive behaviour in the corresponding plans.
%
Let's consider, as a very simple example, yet capturing the point,  
the goal of printing down all the numbers between N and 1,
stopping if/when a 'stop' percept is perceived.
%
In {\asl} this task can be effectively tackled using only g-plans:
%
\begin{small}
\begin{verbatim}
+!print_nums(0).
+!print_nums(_) : stop.
+!print_nums(N) : not stop <- println(N); !print_nums(N-1).
\end{verbatim}
\end{small}
%
\noindent The action \texttt{println} is meant to print the number
on standard output.
%
Here we exploit the fact that the BDI reasoning cycle automatically 
updates beliefs from percepts, and this allows us to write down 
structured plans with courses of actions that change according 
to the environment, by exploiting goals/subgoals and contexts.
%
%This is a clean solution, exploiting the context in plans
%and the fact that in the reasoning cycle beliefs are automatically
%updated given new percepts.
%
More generally, in {\asl} the suggested approach to realise
structured activities whose flow can be environment dependent is by
means of (sub-)goals and corresponding plans with a proper context.
%
In some cases however this approach is not fully effective, and
\emph{e-plans} are needed,
%
in particular every time we need to react \emph{while executing the
  body of a plan}.
%
A typical case occurs when we have actions (or subgoals) in a plan
that could take a long time to complete.
%
For instance, suppose that instead of simply printing we have a
long-term \texttt{elab} goal (it could be even an action):

\begin{small}
\begin{verbatim}
+!print_nums(0).
+!print_nums(N) : stop. 
+!print_nums(N) : not stop <- !elab(N); !print_nums(N-1).
\end{verbatim}
\end{small}

\noindent Suppose that, realistically, we cannot spread/pollute plans
about the goal \texttt{!elab} with a stop-dependent behaviour.
%
To solve this problem, using {\asl} and in BDI architectures
an e-plan can be introduced, using e.g. internal actions to act on the
current ongoing intention:

\begin{small}
\begin{verbatim}
+!print_nums(0).
+!print_nums(N) <- !elab(N); !print_nums(N-1).		
+stop <- .drop_intention(print_nums(_)).
\end{verbatim}
\end{small}
%% RHB: should we change .abort to .drop_intention ?

\noindent The problem here is that the e-plan \texttt{+stop <- ... } is
part of the strategy to achieve the goal \texttt{!print\_nums}, however
it is encoded as a separate unrelated plan.


Finally, the encapsulation of  also impacts  plan failures handling, which is a very
important aspect of agent programming.
%
In the Jason dialect of {\asl}, we can define plans that handle the
failure of the execution of g-plans (generating the event
\texttt{-!g}), but not of e-plans.
%
For example, if there is a problem in the println action in:

\begin{small}
\begin{verbatim}
+stop <- println("stopped").
\end{verbatim}
\end{small}

\noindent the plan execution fails, without any possibility to react
and handle the failure.
%
In order to handle this, a programmer is forced to structure every
e-plan with failure handling using a subgoal:

\begin{small}
\begin{verbatim}
+stop <- !manage_stop.
+!manage_stop <- println("stopped").
-!manage_stop <- println("failure").
\end{verbatim}
\end{small}

\noindent This contributes to making the program longer and verbose,
besides increasing the number of plans to be managed by the
interpreter.

%Besides failure, meta-level actions to manage intentions (plans in execution) at runtime
%works only with g-plans, since they need to specify as argument the specific goal which generated the intention.

\bigskip

To summarise, this is the set of key issues identified for the basic
plan model in the practice of agent programming:
%
\begin{description}
%
\item[Lack of encapsulation:] The strategy to achieve a goal is
  fragmented among multiple plans, not explicitly related to each
  other.
%
\item[Implicit vs. Explicit goals:] Some parts of an agent program may
  have plans for which the goal is implicit.
%
\item[Difficult failure handling:] Failures/errors generated in the
  body of e-plan cannot be directly captured.
%
\end{description}
%%RHB: Are we not talking about the goal-condition and the fact that
%%external events can trigger various plan, one for each intention??



%\noindent This has an impact also on the performance of the agent
%interpreter at runtime. In the traditional model, each reaction
%represented by an e-plan necessarily generates a new intention to
%manage it, increasing the number of plans in execution to be managed
%by the agent reasoning cycle.
%
%This could have also a negative impact on intention selection at
% each reasoning cycle~\cite{DBLP:conf/atal/LoganTY17}.

%\item[Difficult runtime management] -- Intentions generated by e-plans cannot be managed. Besides, as a consequence of point (1), it is not possible to manage (suspend, abort) the intentions related to some goals as a whole.
%
%\item[Runtime overhead] Each reaction (e-plans) is   new intention => new stack. 


\noindent Besides the pure programming perspective, it is worth noting that
these issues can affect the %whose
software engineering process for agent development.
%%
In particular, at the design perspective, it is natural to specify
coarse-grained plans fully encapsulating the strategy to achieve or
maintain goals.
%
It would be important then to keep as much as possible the same level
of abstraction when going from the design to programming, and at
runtime too, to support agent reasoning.


%
%\begin{itemize}
%%
%\item Design / Engineering Drawbacks -- weak encapsulation => impact on modularity => impact on many aspects
%
%%
%\item Reasoning drawbacks -- proliferation of intentions => impact on the reasoning cycle, on intention selection
%%
%\end{itemize}



\section{The {\aser} Proposal}
\label{sec:proposal}

\todo{OLD VERSION}

To overcome the problems discussed above, we consider two key changes
in the plan model.
%
The first one is to extend the plans beyond the simple sequence of
actions and goals, so as to include also the possibility to specify
reactive behaviour encapsulated within the plan.
%
Coherently, with the {\asl} model, such a behaviour can be expressed
in terms of e-plans.
%
% JH: I changed a bit the terminology
Accordingly, a plan becomes the \emph{scope} of ($i$) a sequence of
actions (referred as \emph{body actions}), ($ii$) a set of
\emph{e-plans}, specifying a reactive behaviour which is active at
runtime only when the plan is in execution, and ($iii$) a set of
\emph{g-plans}, specifying plans to achieve subgoals that are relevant
only in the scope of this plan. The e-plans and g-plans are referred
to as \emph{sub-plans}.
%
The sub-plans may include also reactions to failures occurring when
the plan is executed.
%% RHB: So the failure handling plans for the outer goal were moved
%% inside (as well as out)? I'm against this, but I guess this is a
%% democracy :D

%
%
%
The second one is enforcing that e-plans --- as pieces of reactive
behaviours --- must always be defined in the context of a g-plan. We
are thus enforcing the principle that an agent does (and reacts)
always because of a goal to achieve or maintain.
%
This ensures that programmers explicitly specify what is the goal to
achieve even when defining a purely reactive behaviour.
%
In so doing, at runtime every intention\footnote{As in \asl, an
  intention is the result of the deliberation to commit to some
  desire. Briefly, if the agent has an applicable plan for a goal
  event (i.e.\ a desire), it commits to it by creating an intention
  based on that plan and starts executing it.} has an
associated goal being pursued.
%

In the remainder of the section, we first describe in detail the
syntax and informal semantics of the new plan model, including simple
examples, and then we discuss the key benefits.

\subsection{Informal Syntax and Semantics}
\label{sec:infSS}

At the top level, an {\aser} program is a collection of g-plans, whose
syntax is exemplified below:
%
% CURRENT VERSION 0.2 (in the doc)
%
{\small
\begin{verbatim}
/*  g-plans to achieve goal g in context c */
+!g : c <: gc { 

  <- a; b; ?g1; !g1; !!g2. // plan body (optional).

  /* e-plans */
  +e1 : c1 <- b1.	
  +e2 : c2 <- b2.
  +!k : true <: b(10) {
    <- a,b,c.
    +e3 : c3 <- b3. // possible old-style plans
  }	
  
  /* e-plans catching failures */
  -!g[error(ia_failed)] : ...
    <- ... catches from failures 

  /* further g-plans */
  +!g1 : c1 <: gc1 { ... }
}
\end{verbatim}}

Like in {\asl}, a g-plan is defined with a head and a body.
%
% JH: avoid to use the term intention here... goal oriented
% programming. explain intention in the semantics 
Besides the triggering event and the context, in \aser\ the head has a
third new element: a \emph{goal condition}, optionally written after
\texttt{<:}, with the same syntax as the context. While the context is
a pre-condition to select a plan as applicable for an event, the goal
condition is a post-condition that defines when the goal can be
considered as achieved\footnote{Or considered impossible to achieve,
  or the motivation for the goal no longer holds, etc. Programmers can
  use this for any condition that implies the goal should no longer be
  pursued. Note that, when programming declaratively, this condition
  is likely to include the goal in the triggering event itself (as
  believed to be true).}. Any goal created based on this g-plan is
considered achieved if and only if
%% RHB: isn't if and only if too strong?
%% Jomi is probably right, keeping if and only if
this condition holds. If no goal condition is specified, the goal is
considered achieved as soon as the body execution completes, as usual
in \asl. However, if a goal condition is defined, having finished the
body execution is not sufficient to deactivate the goal. Notice that
if the goal condition becomes true while the body actions are being
executed, the execution ceases immediately, since the goal being
achieved means no further action would be necessary.
% aborted. %%AR: isn't 'aborted' suggesting a kind of failure? 
% JH: interrupted is also not but, since it will not be resumed, I changed to stop

In {\aser} the body defines the g-plan scope enclosed by `\texttt{\{}'
and `\texttt{\}}' and is composed of the body actions (after `\texttt{<-}'
and before `\texttt{.}')  as well as sub-plans.
%
Like in {\asl}, as soon as a g-plan is instantiated, the body actions
start to execute. The body actions can be empty --- this is the case
of g-plans expressing a purely reactive behaviour.
%
The sub-plans of the g-plan are considered as relevant only for events
produced by the g-plan execution, since they are in the scope of the g-plan.

% GoalCondition is analogous to the context: however, if context can be considered to be as the pre-condition to execute the plan, the GoalCondition are like a post-condition, which corresponds to a declarative description of the goal to be achieved. 
%
%Alternatively, the execution of a plan can be forced to complete by using an internal action .done - this is useful everytime the goal condition cannot be effectively or naturally described as a boolean expression.
%
%If [GoalCondition] is not specified, by default the condition is that all actions of the main sequence have been executed and completed, in continuity with the AS model. This condition is explicitly represented by the new predefined predicate \texttt{.is\_done}.
%
% The body of the plan provides a lexical and runtime scope of the sub-plans, that is:  variables used in the g/c expression are visible also to subplans the lifetime/availability of the sub-plans is limited to the time in which the g-plan is in execution


While a g-plan is executing, it can be \emph{interrupted} by events
relevant for its e-plans. When the agent perceives an event and an
e-plan from g-plan is applicable according to its specified context,
then the execution of the g-plan body is suspended until the body of
the e-plan finishes its execution.
%
In {\asl}, other plans do also interrupt the execution of a plan in
the case of subgoals. For example, in the body \textsf{a1; a2; !g1;
  a3}, \textsf{g1} is a sub-goal and thus the action \textsf{a3} is
executed after the body of \textsf{g1} has finished. The plan body
execution in interrupted synchronously.
%
In {\aser} this uniformly occurs also with any kind of events, not
only sub-goal events; that is, the body actions can be interrupted to
react to events coming from the environment. However, in this case the
interruption is \emph{asynchronous} -- the point where the body
actions is interrupted is unknown, and depends on the environment, at
runtime.
%
This behaviour is ruled by the (extended) reasoning cycle, which will
be described in next section.


% Failures generated by either the main sequence or by sub-plans generate a -!g that could handled.
%[TO BE COMPLETED?]
%
\subsection{Examples}
To give a more concrete taste of the language, we consider again the
examples seen in Section~\ref{sec:motivation}, now rewritten in
{\aser}.
%
The robot cleaning example becomes:

\begin{small}
\begin{verbatim}
+!clean_env <: false {
   +location(waste,X) : location(robot,X) & location(bin,Y)
      <- pick(waste); !location(robot,Y); drop(waste).
}
\end{verbatim}
\end{small}

\noindent We can give an explicit reason for the reactive behaviour by
encapsulating the e-plan inside a g-plan, with an explicit goal
\texttt{clean\_env}.
%
Setting the goal condition to \texttt{false} means that the plan
execution is going to last forever. It is a way of implementing some
form of maintenance goal.
%
This is also a particular case where the body of the g-plan does not
have any actions.
%

The \texttt{elab\_nums} example seen before can be rewritten to fully
encapsulate the strategy in the same g-plan:

{\small
\begin{verbatim}
+!elab_nums(0).
+!elab_nums(N) <: stop | .body_done {
   <- !elab(N); !elab_nums(N-1).	
   +!elab(M) <- ...
}
\end{verbatim}}

  \noindent Here \texttt{.body\_done} is a predefined internal action
  that succeeds when the body actions have been completed. The goal
  \texttt{elab\_nums} is achieved either by the perception of
  \texttt{stop} or by the execution of its body actions.
%
  In case we want to take some action to react to \texttt{stop}, we
  can introduce an e-plan as follows:

{\small
\begin{verbatim}
+!elab_nums(0).
+!elab_nums(N) {
   <- !elab(N); !elab_nums(N-1).
   +!elab(N) <- ...
   +stop <- println("stopped"); .done.
}
\end{verbatim}}

  For the example of plans to handle social obligations, we can define
  an explicit goal by encapsulating the e-plans in a g-plan as
  follows:
\begin{small}
\begin{verbatim}
+!be_obedient <: false {
   +obligation(Ag,committed(Goal)) : .my_name(Ag)      <- ...
   +obligation(Ag,achieve(Goal))   : .my_name(Ag)      <- ...
   -obligation(Ag,Goal) : .my_name(Ag) & .intend(Goal) <- ...
   ...
}
\end{verbatim}
\end{small}
\noindent Now the agent can disable these plans by simply performing
\texttt{. drop\_intention(be\_obedient)}, and resuming its obedient
behaviour by adopting the goal \texttt{!be\_obedient}, and it can also
check whether it is being obedient by testing
\texttt{.intend(be\_obedient)}.

With the new language, we can program different kinds of commitments
to goals~\cite{cohen:90,winikoff:02} by exploiting the goal
condition. For instance, the Single Minded Commitment for goal
\texttt{g} can be programmed as:
\begin{small}
\begin{verbatim}
+!g <: g | f {
   ...
}
\end{verbatim}
\end{small}
where \texttt{f} states when the goal becomes impossible. The agent
commits to achieve \texttt{g} until \texttt{g} is believed either true
or impossible to achieve. To program this kind of commitment in \asl,
we have to follow some programming patterns requiring extra plans
(three extra plans as shown in~\cite{hubner:06b}).




%%%RHB: The paragraph below isn't clear to me
%  \bigskip The proposed approach allows us to avoid towers of subgoal
%  calls, that are used when a plan does some action and then should
%  wait that some condition is achieved, in order to complete,
%  eventually reacting to some events to adapt the course of actions.
%%
%An example is a simple thermostat. 
%%
%Let's consider the goal to achieve some temperature +!temp(T):
%
%{\small
%\begin{verbatim}
%+!temp(T) : temp(CT) & CT < T & not warming
%  <-  switchOnWarming; !temp(T).
%+!temp(T) : temp(CT) & CT > T & not cooling
%  <-  switchOnCooling; !temp(T).
%+!temp(T) : temp(CT) & CT != T 
%  <-  !temp(T).
%+temp(T) : temp(T) <-  stop.
%\end{verbatim}}
%
%  \noindent There is here a kind of \emph{cognitive busy waiting}
%  situation: after switching on the HVAC system (to cool down or warm
%  up), we need to check continuously the condition to
%  know when/if the goal has been achieved, before ending the plan.
%%
%  This is problematic for 2 reasons: the size of the intention stack
%  is growing indefinitely and performance overhead.
%% JH: not sure we should go for performance arguments as above. the
%% need to check all conditions every cycle is equivalent to the cost
%% of the above cognitive busy waiting
%% RHB: Furthermore, is the intention stack really growing indefinitely
%% in this case??? Probably not as it's just tail recursion and Jomi
%% optimised that in Jason, no need to use !! anymore. 
%%
%The \aser version avoids that problem:
%
%{\small
%\begin{verbatim}
%+!temp(T) : temp(CT) & CT < T 
%  <: temp(T) & not warming & not cooling {  
%  <- switchOnWarming.
%  +temp(T) <- stop. // There is bound
%}
%
%+!temp(T) : temp(CT) & CT > T 
%  <: temp(T) & not warming & not cooling {  
%  <- switchOnCooling.	
%  +temp(T) <- stop.		
%}
%+!temp(T) : temp(T).
%\end{verbatim}}
%
%% JH: I am not sure to present this example, seems more complex than original code
%% JH: it is not equivalent to ASL version: if the first plan is
%% triggered, how could the second be triggered latter?
%% RHB: Personally, I don't like examples about thermostats very much...
%
%Remarks:
%\begin{itemize}
%\item the T variable used inside the plan body (+temp(T) <-..) has been bound in +!temp(T)
%\end{itemize}
%
%\noindent  In Jason, the original AgentSpeak(L) solution could be avoided  by exploiting a .wait internal action that suspends a plan until some condition is achieved:
%
%{\small
%\begin{verbatim}
%+!temp(T) : temp(CT) & CT < T 
%  <-  switchOnWarming;
%      .wait(temp(T));
%      stop.
%
%+!temp(T) : temp(CT) & CT > T 
%  <-  switchOnCooling;
%      .wait(temp(T));
%      stop.
%\end{verbatim}}
%
%  \noindent However, this solution is no longer effective (by itself)
%  if something could happen while waiting which would require changing
%  the course of action.
%%
%  For instance, suppose that while warming up, the temperature value
%  surpasses the target value T, such that we would need to start
%  cooling. In this case, the .wait solution must be strongly reworked
%  and becomes complicated.
%%
%The complete solution using the extended plan model is as follows:
%
%{\small
%\begin{verbatim}
%+!temp(T) <: temp(T) & not warming & not cooling {
%  +temp(T) <- stop.		
%  +temp(T1) : T1 > T <- switchOnCooling.
%  +temp(T1) : T1 < T <- switchOnWarming.
%}
%\end{verbatim}
%}
%
%% JH: shouldn't we present only the above solution? clear and simpler 
%% is the test not warming/ not cooling necessary on the GC
%
%% RHB: The goal condition of the plan above is strange. What is if
%% for? Isn't it the same as .done in the first +temp plan?
%
%\noindent Maintenance goal: to achieve and keep the temperature at the target temperature T, with the possibility of dynamically updating/changing T.
%% RHB: The phrase above needs improvement
%
%{\small
%\begin{verbatim}
%+!target_temp(T) <: false {
%  <- +target_temp(T).	
% 
%  +temp(T) : target_temp (T) <- stop.		
%  +temp(T1) : target_temp(TT) & T1 > TT <- switchOnCooling.	  
%  +temp(T1) : target_temp(TT) & T1 < TT <- switchOnWarming.
% 
%  +target_temp(TT) : temp(TT)  <- stop.	
%  +target_temp(TT) : 
%    temp(T) & T < TT & not warming <- switchOnWarming.
%  +target_temp(TT) : 
%    temp(T) & T > TT & not cooling <- switchOnCooling.
%}
%\end{verbatim}}

 \subsection{Failure Management}

 Failures generated by the execution of actions of sub-plans for a
 goal \texttt{g} can be handled and managed by:

%%RHB: Carefull, changed but feel free to 
\begin{itemize}
\item subplans of type \texttt{-!g} listed in the body of the g-plan;
\item plans at the same level of \texttt{+!g}, if the event is not
  managed within the body.
\end{itemize}

\noindent In terms of style, it is more in keeping with the new
language to include the failure plans within the scope of the
g-plan. However, it is also true that if we define a plan for
\texttt{+!g} at a certain level of a g-plan tree\footnote{Note that
  g-plans within g-plans now implicitly form a plan tree for a
  top-level goal, and the plan library is thus a forest of such
  trees.}, it might make more sense to the programmer if the goals for
\texttt{-!g} are all placed at the same level (specially if the
programmer is influenced by the style of the Jason variant of
AgentSpeak).

% \noindent In the latter case, the event representing the failure could
% be exactly \texttt{-!g}, as in Jason, i.e. the event which is
% generated when the goal is removed.
%
% This strategy cannot be adopted in the former case, since we want to
% react to a failure but the corresponding goal and intention has not
% been removed yet, so no -!g event is generated. For this reason, we
% need to introduce a new event: \texttt{+/err\_term} [TEMPORARY SYNTAX]
% representing the failure of an action belonging to the main sequence
% or to the body of subplans.
%%
%% RHB: I find this rather clumsy, besides not being formalised. Is it
%% worth going into details of a new mechanism for plan failure? We
%% could just claim it's as before plus now they can be inside the
%% g-plan IF YOU MUST :D
%% Besides, I can't see why we can't generate a -!g event even without
%% removing the rest of the intention (that's what Jason always did
%% anyway).

%% RHB: I'm changing this as discussed with Jomi but feel free to
%% bring it back
% {\small
% \begin{verbatim}
% +!g : c <: GoalCondition {
%   <- a; b; ? ; c. 
%   ...
%   +/err_term : cn <- bn
% }
% \end{verbatim}}

%   \noindent In the reasoning cycle, the \texttt{+/err\_term} is
%   treated as the other events: if an applicable plan is found, it is
%   selected and executed (in the same intention).
% %
%   Otherwise, the behaviour is like it is in pure AgentSpeak(L) --- the
%   whole intention is removed and a \texttt{-!g} event is generated.

%\subsection{Discussion}
% [TO BE COMPLETED]
%
% In this section we go back on the key issues, showing how the approach is effective in overcoming them.
%
\subsection{Key points}

Before giving a more formal description of the semantics, we conclude
this section by summarising the key points brought out by {\aser}:
%
\begin{description}
%
\item[Encapsulation] -- The strategy to achieve a goal is encapsulated
  in one or multiple g-plans, each embedding also the reactive
  behaviour which is part of the strategy.
%
  The effect is to reduce code fragmentation, improving its
  understanding.
%
\item[Explicit goals] -- Every behaviour of the agent is now
  explicitly related to some goal to achieve.
%
  This promotes a more goal-oriented programming style, yet preserving
  the possibility to easily define g-plans based on purely reactive
  strategies;
%
  this allows the agent to better \emph{manage} its intentions. For
  instance, a programmer can now do a simple
  \texttt{.drop\_intention(g)} to disable the behavior of e-plans
  embedded in the corresponding g-plans.
%
% In {\asl} this is quite tricky to do.
In {\asl}, the relation goal-intention is not one-to-one. {\asl} can have intentions from e-plans and so intentions without a explicitly represented goal. 
In {\aser} there is a one-to-one relation between goals and intentions. 
%
All intentions come from goals (only explicit goals are allowed). 
%
Primitives to handle goals can thus handle \emph{all} intentions. 
%
In {\asl}, primitives to handle goals can manage a limited set of the
intentions (those created from goal addition plans).


\item[Failure handling] -- Thanks to explicit goals, failures
  generated in the body of an e-plan can now be directly captured and
  managed by failure recovery plans defined for g-plan, without the
  need for auxiliary goals and plans to be introduced.

\item[Coarse-grained intentions] -- In {\asl}, each e-plan in
  execution has its own intention/stack, which runs concurrently to
  the other intentions. Conceptually, this follows the idea that
  the management of environment events are not part of an existing
  overall plan to achieve some goal.
%
  In the new model instead, an e-plan inside a g-plan is meant to
  specify a behaviour that is useful for achieving the goal of the
  g-plan, so part of the same intention.
%
  For this reason, if an e-plan (sub-plan) inside a g-plan in
  execution is triggered, no new intentions are generated and the body
  of the sub-plan is placed on the top of the stack of the same
  intention.
%
  The new model leads then to more coarse-grained intentions.
%
%%% JH: commented for now, while we are discussing it
%The general effect is to reduce the number of intentions. 
%
%This has an impact on the reasoning cycle, for instance on the step selecting the intention to carry on.
\end{description}
%\subsubsection{Intention implicit Interruption}
%
%The effect is to implicitly interrupt and suspend the main sequence (body) of the g-plan - if available and if not already completed.
%%
%Conceptually, this is the asynchronous version of the suspension that occurs when a subgoal is specified (!g) in the body of a plan. This causes exactly the suspension of the body and the pushing of the body of the new plan on the top of the stack of the intention.

% \subsubsection{Compatibility with {\asl}}


\section{Formal Syntax and Semantics}
\label{sec:formalisation}

We first give the formal syntax of \aser, as per the following
grammar:

\begin{figure}[htbp]
  \footnotesize{\ttfamily
    \begin{center}
      \begin{tabular}{l c l l} 
   ag & ::= & bs gs gps \\
   bs & ::= & af$_1$ ... af$_{\textrm{n}}$   & (n $\geq$ 0)\\
   gs & ::= & \textbf{!}af$_1$ ... \textbf{!}af$_{\textrm{n}}$ & (n $\geq$ 0)\\
   gps & ::= & gp$_1$ ... gp$_{\textrm{n}}$  & (n $\geq$ 1)\\
   gp & ::= & \textbf{+!}af \textbf{:} lf \textbf{<:} lf \textbf{\{} [
              \textbf{<-} h\textbf{.} ] ( gp \galt op )* \textbf{\}}
              \\ % g-plans 
   op & ::= & te \textbf{:} lf \textbf{<-} h\textbf{.} \\ % old style plans
   te & ::= & \textbf{+}af \galt \textbf{-}af \galt \textbf{+}g  \galt
              \textbf{-}g \\ 
   g  & ::= & \textbf{!}af \galt \textbf{?}af \\
   af & ::= & P\textbf{(}t$_1$\textbf{,} \ldots \textbf{,}
              t$_n$\textbf{)} & (n $\geq$ 0)\\   
   lf & ::= & af \galt $\neg$ lf \galt (lf $\wedge$ lf) \galt (lf $\vee$
              lf) \galt \texttt{true} \\
   h  & ::= & h$_1$\textbf{;} \texttt{true} \galt \texttt{true} \\
   h$_1$ & ::= & a \galt g \galt h$_1$\textbf{;} h$_1$ \\
   a  & ::= & A\textbf{(}t$_1$\textbf{,} \ldots \textbf{,} t$_n$\textbf{)} &  (n $\geq$ 0) \\
\end{tabular}
    \end{center}
}
\caption{Syntax of \aser}
\label{fig:syntax}
\end{figure}

\noindent
where metavariable $P$ stands for a predicate symbol, $t_i$ stand for
first order terms (as in Prolog for example), and metavariable $A$
stands for an action symbol. Note also that this is an abstract
syntax, it contains only the fundamental aspects of the language for
the purposes of formalisation. Those familiar with the actual systems
where \aser has been implemented should be aware that all the
syntactic features of the original systems are of course kept. For
example, the syntax here does not include the \verb|!!| operator,
annotations, and many other \jason features.

%% First update reasoning cycle states...

Briefly, these are the changes in the semantics of \asl
that are needed to accomodate the new features of \aser:

\begin{itemize}
\item in the beginning of the reasoning cycle, check all
  goal-conditions starting from the bottom of each intention stack and
  remove all finished intentions;

\item if an event is \emph{external}, try to trigger one
  applicable plan for each intention, starting from the
  top of the intention stack (that is, the most specific plan in the
  respective g-plan tree);

\item if an event is \emph{internal}, search for an applicable plan on
  the path to the root of the g-plan tree, that is, starting from the
  most specific plans in the g-plan tree (as in the item above but now
  there is only one intention to refer to); this does not change the
  semantics per se, only the way in which we look for relevant
  applicable plans.
\end{itemize}

We start by updating the well-known reasoning cycle~\cite{bordini:07}
to adapt it for the initial stage of checking all goal-conditions to
remove the intentions that should no longer be active. Furthermore,
note that even though this is computationally costly, it is necessary
because not doing it at every reasoning cycle may imply the agent
missing the moment where the goal-condition became true (hence an
intention needing to be deactivated); it is also worth the
computational burden in as much as it has the various practical
programming advantages we pointed out earlier in this paper.

%% RHB: I removed this text from above, but I guess it should be in
%% the initial sections
% Recall that this can be because the goal has been achieved, has become
% impossible to achieve, or the motivation why it was adopted no longer
% applies.

The required stage for checking goal-conditions is included in the
existing $\ClrInt$ stage (which previously only removed empty
intentions) except it is now moved to the beginning of the reasoning
cycle (to ensure nothing in the reasoning cycle is done under the
assumption a deactivated intention is still active), just after the
$\ProcMsg$ stage (as the information just received from other agents
might be useful in checking for goals to be deactivated). The clearing of
intentions used to be the last part of the reasoning cycle, but
because there are no other dependencies between the first and last
stages, $\ClrInt$ might as well be done at the beginning rather than
the end. The slightly changed reasoning cycle is shown in
Figure~\ref{fig:rcaser}.

\begin{figure}[htbp]
  \begin{center}
    \includegraphics[width=\linewidth]{figs/ASERrc.pdf}
    \caption{The AgentSpeak(ER) Reasoning Cycle}
    \label{fig:rcaser}
  \end{center}
\end{figure}

The $\ClrInt$ stage also needs new semantic rules. In fact, the
previous 3 \rn{ClrInt} rules (see~\cite[p~212]{bordini:07}) are no
longer needed, as empty intentions are not to be removed from the set
of intentions, unless their goal condition becomes true. To facilitate
the presentation of the new rules, we define a new auxiliary function
$\GCOND$, which given a g-plan simply returns the goal-condition
component of that plan, as follows. Let $p=$``\texttt{+!g : c <: gc
  \{<- a.\}}'', then $\GCOND(p)=\texttt{gc}$, more specifically the
logical formula coded by \texttt{gc}. Also, we denote by
$[p_1,...,p_n]$ an intention stack with plan $p_n$ at its top.

The new rule \rn{ClrInt$_1$} is as shown below, and rule
\rn{ClrInt$_1$} is not shown because it simply causes the transition
$\CFG{\ClrInt} \trans \CFG{\SelEv}$ in case the negation of the
precondition of \rn{ClrInt$_1$} holds. The rule below essentially
removes from intentions the bottom-most g-plan for which its
goal-condition now follows from the state of the belief base. The
intuition is that if goal $\texttt{!g1}$ required a subgoal
$\mathtt{!g2}$ (a plan for which was pushed on top of the plan for the
former one in the intention stack) and $\texttt{!g1}$ is no longer
active, that whole part of the intention stack above it needs to be
removed together with it.

\infrule[ClrInt$_1$] {
        i \in \CI \qquad i=[p_1,\dots,p_n] \\
        \AGBELS \models \GCOND(p_j) \mbox{ for some } j, 1\leq j\leq n
}
%------------------------------------------------
{   \CFG{\ClrInt}  \trans  \CFGcp{\ClrInt} \\[1.2mm]  
\begin{array}{llcl}
  \mbox{\emph{where:}\quad}
     & \multicolumn{3}{l}{\mbox{$j$ is the least number in [1..n]
       s.t.}}\\
     & & & \AGBELS \models \GCOND(p_j)\\
     & \CIli & = & \CI \setminus \{i\} \cup \{[p_1,\ldots,p_{j-1}]\} \\
\end{array}
}

The auxiliary functions $\RELPLANS(\PLANS,\TE)$
$\APPLPLANS(\PLANS,\TE)$ (see Definitions~10.2 and~10.3
in~\cite{bordini:07}) can be changed to accommodate both the new g-plan
structure as well as the firing of e-plans (i.e., plans for reacting
to external events) for all intentions rather than creating a single
new separate intention as before. First, the redefined functions now
receive a set of intentions as an extra parameter; in the new semantic
rules they are called with the agent's plan library and the current
contents of the set of intentions ($\AGPLANS$ and$\CI$ in the
operational semantics, respectively). If $\TE$ is an external event,
the new parameter is used to check for applicable plans for each
individual intention plus the empty intention\footnote{The empty
  intention being included in the set of intentions in that parameter
  is useful for backwards compatibility with traditional AgentSpeak(L)
  but we do not discuss this here as it would hinder the explanation
  of the essential aspects of the proposed extension.}
$\EXT$. Besides having an extra parameter, the functions now return a
set of triples rather than a pair. We use $\TREV(p)$ to refer to the
triggering event of a plan $p$, and $\SCOPE([\PL_1,\ldots,\PL_n])$
returns all plans in the scope of the g-plan $p_n$. By ``in scope'',
we mean the plans that appear (immediately) within the braces
delimiting a g-plan (but not within its subgoals); for example the
plans for \texttt{+e1}, \texttt{+e2}, and \texttt{+!k} (but not for
\texttt{+e3}) are in the scope of the g-plan \texttt{+!g : c <: gc \{
  $\ldots$ \}} shown in the beginning of
Section~\ref{sec:infSS}. Furthermore, note that the $\SCOPE$ function
can determine the exact plan for $\PL_n$ in the forest of g-plan trees
now forming the plan library by using plans
$\PL_1,\ldots,\PL_{n-1}$. We can now formally define $\RELPLANS$:

\begin{definition}[Relevant Plans]\label{def:relplans}
  The auxiliary function to retrieve relevant plans given a plan
  library $PL$, a particular event $\TE$ of type \texttt{+l} or
  \texttt{-l} (i.e., an external event, one reacting to changes in
  beliefs rather than a goal adoption event), and a set of intentions
  $I$ is defined as
  $\RELPLANS(PL,\TE,I) = \{ \langle\PLANS,\theta,i\rangle \mid i \in
  I, i = [\PL_1,\ldots,\PL_n],$
  and
  $ps = \{ rp \mid rp \in \SCOPE([\PL_1,\ldots,\PL_j]) \mbox{ and }
  \{\TE\} \models \TREV(rp)\theta$,
  for some m.g.u.\ $\theta\}$, where $j$ is the greatest number in
  $[1..n]$ s.t.\ $ps \neq \emptyset$, or $ps=\emptyset$ if there is no
  such $j\}$. For internal events, the function returns a set
  containing a single element $\langle\PLANS,\theta,i\rangle$ where
  $i$ is the specific intention that generated $\TE$ and $\PLANS$ is
  the set of relevant plans using the same idea of scope as above for
  external events.
\end{definition}

We do not formally define the $\APPLPLANS$ function here due to space,
and given that it is trivially extended in a very similar way as in
Definition~\ref{def:relplans} for $\RELPLANS$.

Finally, we need to change the semantics so that not just one but
\emph{all} relevant e-plans are triggered, that is, one for
each active intention, and within a single intention starting from the
most specific plan (i.e., the one closest to the top of the intention
if there are relevant plans at other levels too). Most of the work was
already done in the redefinitions of the $\RELPLANS$ and $\APPLPLANS$
functions, but we still need to change the rule for processing
external events (which now change multiple intentions). Note however
that by the time we come to handle an external event with rule
\rn{ExtEv}, a single plan for each intention has already been
selected; that is, $\SO$ (the option selection function) is also
slightly redefined to work with the new structures returned by those
redefined auxiliary functions.

The new \rn{ExtEv} rule is below, and it essentially says that an
external event now potentially interrupts various intentions, provided
there are relevant and applicable plans anywhere in the g-plans
associated with each particular current intention of the agent. The
$\TRHO$ component of the transition system configuration has the
result of the application of the new $\RELPLANS$ and $\APPLPLANS$
functions.
% Recall that external events for the ``main'' goal --- the external
% events as in traditional AgentSpeak --- will now be associated with
% the empty intention \EXT in \TRHO (i.e., the result of the
% $\RELPLANS$ and $\APPLPLANS$ functions).
The chosen plan for each intention, after being applied its respective
variable substitution $\theta$, is pushed on top of that intention
(the notation used for this below is $i_j[\PL_j\theta_j]$).

\infrule[ExtEv] {
        \TEPS = \langle\TE,\EXT\rangle \qquad \TRHO =
        \{\langle\PL_1,\theta_1,i_1\rangle,\ldots,\langle\PL_n,\theta_n,i_n\rangle\}
}
%------------------------------------------------
{   \CFG{\ClrInt}  \trans  \CFGcp{\ClrInt} \\[1.2mm]  
\begin{array}{llcl}
  \mbox{\emph{where:}\quad}
     & \CIli & = & \CI \setminus \bigcup_{j=1}^{n} \{i_j\} \cup
                   \bigcup_{j=1}^{n} \{i_j[\PL_j\theta_j] \}\\
\end{array}
}

To conclude this section, we emphasise that, for the sake of space, we
here formalised only the main changes to the well-known AgentSpeak
semantics. The complete new transition system giving semantics to
AgentSpeak(ER) will appear in a longer paper.

\section{Evaluation}
\label{sec:evaluation}

\todo{
- Implementation of the model using two main platforms: Jason and ASTRA \\
- Discussion (about benefits, performance penalty, etc) \\
}

%\begin{itemize}
%\item general discussion about how the new feature impact on the BDI reasoning cycle
%\item implementation in Jason
%\begin{itemize}
%\item syntax and general "feelings"
%\item performance impact
%\end{itemize}
%\item implementation in ASTRA
%\begin{itemize}
%\item syntax and general "feelings"
%\item {performance impact}
%\end{itemize}
%\end{itemize}


\section{Related Work}
\label{sec:related}

\todo{OLD}

{\aser} is primarily related to work in literature focusing on improving cognitive BDI agent programming.
%
% MODULARITY
%
A main aspect widely discussed and developed in the literature is  \emph{modularity}~\cite{Madden2010, Busetta2000,5285116,Novak:2006:MBA:1160633.1160814,Ortiz-Hernandez2016,vanRiemsdijk:2006:GMA:1160633.1160864,Hindriks2008,Nunes2014}.
%
In programming languages, modularity is strongly related to encapsulation, in fact  strengthening encapsulation typically leads to refining modularity, in particular devising more coarse-grained modules.
%
{\aser} enriches the spectrum of approaches elaborated in literature for improving modularity in BDI-based agent programming languages by devising coarse-grained plans as modules encapsulating goal-oriented \emph{and} reactive behaviour. 

Besides, {\aser} is related to existing BDI agent programming languages extending the basic plan model as found in the original proposal of {\asl}.
%
In this context, a main reference is \textsf{CANPlan}~\cite{Sardina2011}, a  BDI-style agent-oriented programming language enhancing usual BDI programming style with declarative goals, look-ahead planning, and failure handling. 
%
It allows programmers to mix both procedural and declarative aspects of goals, enabling reasoning about properties of goals and decoupling plans from what these plans are meant to achieve. 
%
The lookahead planning makes it possible to guarantee goal achievability and avoid undesirable situations. 
%
The plan model adopted in {\textsf{CANPlan}} is analogous to the {\asl} one. Each plan is characterised by a plan rule \textsf{e(t) : $\psi$(xt, y) $\leftarrow$ P(xt, y, z).}, where \textsf{P} is a ``reasonable strategy'' to follow when \textsf{$\psi$} is believed true in order to resolve/achieve the event.
%
\textsf{P} can be a rich composition of actions but not reactions.  
%
Reactive behaviours can be expressed instead --- like in {\asl} and in the basic BDI --- as separate plans handling belief updates corresponding  to environment events. 

%% RHB: OK to change this title as below?
%\section{Conclusion and Future Work}
\section{Conclusion}
\label{sec:conclusion}

\todo{OLD, TO BE REWORKED}

In this paper, we introduced \aser, a novel extension of the classical
{\asl} language. The language provides encapsulation for agent
goals, which clearly improves legibility and reusability of AgentSpeak
code. Furthermore, the new language improves some of the shortcomings
of AgentSpeak in regards to goal orientation and declarative goals by
ensuring that all reactive plans are also associated with general
goals, providing a ``goal condition'' which means goals can be still
active even though presently there is no action for the agent to take
towards that goal, and allowing external events (i.e., reactions to
changes in beliefs) to trigger various plans, for all the goals it
might be relevant.  We formalised the main changes required in the
existing formal semantics of AgentSpeak and experimentally evaluated
an interpreter for \aser\ implemented on top of the ASTRA platform.


%% RHB: OK to mention the multi-agent contest?

% \begin{itemize}
% \item refining current ASTRA implementation, that will allow to refine the evaluation
% \item Jason extension implementing {\aser}, and compare it with the ASTRA one
% \item stress {\aser} in practice 
% \item think about the tools: how existing IDE and tools can be extended to provide functionalities exploiting the new features
% \item investigate how the extension either improve or not the possibility to formally analyse the correctness properties
% \item ...
% \end{itemize}




%\begin{acknowledgements}
%If you'd like to thank anyone, place your comments here
%and remove the percent signs.
%\end{acknowledgements}

% BibTeX users please use one of
\bibliographystyle{spbasic}      % basic style, author-year citations
%\bibliographystyle{spmpsci}      % mathematics and physical sciences
%\bibliographystyle{spphys}       % APS-like style for physics
\bibliography{main}  % put name of your .bib file here

\end{document}
% end of file template.tex

